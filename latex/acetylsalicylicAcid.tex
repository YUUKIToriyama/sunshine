\documentclass[a3paper]{ltjsarticle}
\usepackage{chemfig}
\usepackage[version=4]{mhchem}
\begin{document}
\title{アセチルサリチル酸の合成反応}
\author{YUUKIToriyama}
\date{}
\maketitle

%% フェノール
%% \chemfig{*6(=-=(-OH)-=-)}
%% アセチル酸ナトリウム
%% \chemfig{*6(=-=(-ONa)-(-C(=[:150]O)-[:30]ONa)=-)}
%% アセチル酸
%% \chemfig{*6(=-=(-OH)-(-C(=[:150]O)-[:30]OH)=-)}
%% 無水酢酸
%% \chemfig{H_3C-[:30](=[:90]O)-[:-30]O-[:30](=[:90]O)-[:-30]CH_3}
%% アセチルサリチル酸
%% \chemfig{*6(=-=(-O-[:-30]C(=[:-90]O)-[:30]CH_3)-(-C(-[:30]OH)=[:150]O)=-)}
%% 酢酸
%% \chemfig{CH_3-[:30](=[:90]O)-[:-30]OH}


	\ce{
		\chemfig{*6(=-=(-OH)-=-)} + CO_2 + 2NaOH -> \chemfig{*6(=-=(-ONa)-(-C(=[:150]O)-[:30]ONa)=-)} + 2H_2O
	}
	

	フェノールを高温高圧下で二酸化炭素と水酸化ナトリウムと反応させてサリチル酸ナトリウムを合成する。
	\vspace{20pt}


	\ce{
		\chemfig{*6(=-=(-ONa)-(-C(=[:150]O)-[:30]ONa)=-)} + 2H_2SO_4 -> \chemfig{*6(=-=(-OH)-(-C(=[:150]O)-[:30]OH)=-)} + 2NaHSO_4
	}

	さらに、この二ナトリウム塩を希硫酸で中和し、サリチル酸を遊離させる。
	\vspace{20pt}


	\ce{
		\chemfig{*6(=-=(-OH)-(-C(=[:150]O)-[:30]OH)=-)} + \chemfig{H_3C-[:30](=[:90]O)-[:-30]O-[:30](=[:90]O)-[:-30]CH_3} -> \chemfig{*6(=-=(-O-[:-30]C(=[:-90]O)-[:30]CH_3)-(-C(-[:30]OH)=[:150]O)=-)} + \chemfig{CH_3-[:30](=[:90]O)-[:-30]OH}
	}
	
	このサリチル酸に無水酢酸を作用させてアセチル化し、アセチルサリチル酸を得る。

\end{document}
