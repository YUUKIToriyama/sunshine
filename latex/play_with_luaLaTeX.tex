\documentclass[a4paper]{article}
\usepackage{luatexja}
\usepackage{luacode}

\usepackage{tikz}
\usepackage{color}

\begin{luacode*}
	-- LaTeXの改行を出すための函数
	function n()
		return string.char(92)..string.char(92)
	end
	
	-- パスカルの三角形を計算するスクリプト
	function pascalTriangle(row, upto)
		tex.sprint(table.concat(row, " ")..n())
		if #row < upto then
			local i
			local tmp = {1}
			for i = 1, (#row - 1) do
				table.insert(tmp, row[i] + row[i+1])
			end 
			table.insert(tmp, 1)
			pascalTriangle(tmp, upto)
		end
	end

	-- 円周率を計算するスクリプト(モンテカルロ法)
	function calcPi(m)
		local count = 0
		for i = 1, m do
			x, y = math.random(), math.random()
			if x^2 + y^2 < 1 then
				count = count + 1
			end
		end
		return {m,count, 4 * count / m}
	end

	-- 素数を計算するスクリプト
	function primeSeries(m)
		print(1)
		if m <= 1 then
			tex.sprint("invalid argument")
		else
			for i=2,m do
				flag = true
				for n=2,math.ceil(math.sqrt(i)) do
					if i % n == 0 and i ~= n then
						flag = false
						break
					end
				end
				if flag then
					tex.sprint("\\textcolor{red}{"..i.."} ")
				else
					tex.sprint(i.." ")
				end
			end
		end
	end
\end{luacode*}

\begin{document}

\title{luaLaTeXであそぼう}
\author{YUUKIToriyama}
\date{}
\maketitle

	\centering
	\directlua{pascalTriangle({1}, 17)}	

	\vspace{49pt}
	\begin{tikzpicture}
		\draw[thick, -stealth](-1,0)--(6,0) node [anchor=north]{$x$};
		\draw[thick, -stealth](0,-1)--(0,6) node [anchor=east]{$y$};
		\draw[domain=0:5] plot(\x, {sqrt(25 - \x^2)});
		\node [anchor=north] at(3,3){$x^{2}+y^{2}=1$};
		\draw [thick](0,5) node [anchor=east]{$1$}--(5,5)--(5,0) node [anchor=north]{$1$};
		\node [anchor=south] at(5.5,5){$(1,1)$};
	\end{tikzpicture}

	\directlua{
		local result = calcPi(50000)
		tex.print("試行回数: " .. result[1] .. n())
		tex.print("円の中に入った回数: " .. result[2] .. n())
		tex.print("円周率: " .. result[3] .. n())
		tex.print("誤差: " .. 1 - math.pi / result[3] .. n())
	}
	
	\vspace{49pt}
	
	\directlua{
		local num = 999
		tex.sprint(num.."までの素数を赤文字で表示"..n())
		primeSeries(num)
	}
	
	
\end{document}
