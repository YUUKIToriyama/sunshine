\documentclass[a4paper]{article}
\usepackage{luatexja}
\usepackage{luacode}

\usepackage{graphicx}
\usepackage{tikz}
\usepackage{color}
\usepackage{amsmath}
\usepackage{ascmac}

\begin{luacode*}
	-- LaTeXの改行"\\"を出すための函数
	function n()
		return string.char(92)..string.char(92)
	end

	-- 現在の時刻を呼び出すスクリプト
	function now()
		d = os.date("*t", t)
		return d.year .. "/" .. d.month .. "/" .. d.day .. " " .. d.hour .. "時" .. d.min .. "分" .. d.sec .. "秒"
	end

	
	-- パスカルの三角形を計算するスクリプト
	function pascalTriangle(row, upto)
		tex.sprint(table.concat(row, " ")..n())
		if #row < upto then
			local i
			local tmp = {1}
			for i = 1, (#row - 1) do
				table.insert(tmp, row[i] + row[i+1])
			end 
			table.insert(tmp, 1)
			pascalTriangle(tmp, upto)
		end
	end

	-- 円周率を計算するスクリプト(モンテカルロ法)
	function calcPi(m)
		local count = 0
		for i = 1, m do
			x, y = math.random(), math.random()
			if x^2 + y^2 < 1 then
				count = count + 1
			end
		end
		return {m,count, 4 * count / m}
	end

	-- 素数を計算するスクリプト
	function primeSeries(m)
		print(1)
		if m <= 1 then
			tex.sprint("invalid argument")
		else
			for i=2,m do
				flag = true
				for n=2,math.ceil(math.sqrt(i)) do
					if i % n == 0 and i ~= n then
						flag = false
						break
					end
				end
				if flag then
					tex.sprint("\\textcolor{red}{"..i.."} ")
				else
					tex.sprint(i.." ")
				end
			end
		end
	end

	-- 平方根を計算するスクリプト
	function calcSqrt(n,times)
	        local a = n/2 
	        local b = 2 

	        for i=1,times do
	                a,b = (a+b)/2, 2*a*b/(a+b)
	        end

		return {a,b}
	end

	-- socket.httpを用いて画像をダウンロードするスクリプト
	function getRemoteImage(url)
		local io = require("io")
		local http = require("socket.http")
		local md5 = require("md5")

		math.randomseed(os.time())
		local id = md5.sumhexa(math.random())..".jpg"
		local res = http.request(url)
		io.output(id)
		io.write(res)
		io.close()

		tex.print(string.char(92).."includegraphics{"..id.."}")
	end

\end{luacode*}

\usepackage{listings}
\lstset{
	language={[5.3]Lua},
	backgroundcolor={\color[gray]{.90}},
	breaklines=true,
	commentstyle={\ttfamily \color[rgb]{0,0.5,0}},
	keywordstyle={\bfseries \color[rgb]{0,0,1}},
	stringstyle=\ttfamily,
	columns=fullflexible,
	numbers=left,
	numberstyle={\tiny \emph},
	numbersep=5pt
}

\begin{document}

\title{luaLaTeXであそぼう}
\author{YUUKIToriyama}
\date{}

\maketitle

	\section{はじめに}
	
	\subsection{このドキュメントについて}
	このドキュメントはluaLaTeXでコンパイルすることによって見ることができます。このファイル内に書かれたプログラムをLuaインタープリタが解釈し、その結果を埋め込んだ文章を出力します。

	たとえば、次のようなスクリプトが\symbol{92}begin\symbol{123}luacode*\symbol{125}〜\symbol{92}end\symbol{123}luacode*\symbol{125}内に書かれています。

	\begin{lstlisting}
-- 現在の日時刻を"year/month/day hour時min分sec秒"の形式で返す関数
function now()
	d = os.date("*t", t)
	return d.year .. "/" .. d.month .. "/" .. d.day .. " " .. d.hour .. "時" .. d.min .. "分" .. d.sec .. "秒"
end
	\end{lstlisting}

	これを\symbol{92}directlua\symbol{123}now()\symbol{125}によって呼び出すと次のようになります。表示されない場合はluaLaTeXを用いていないか、環境が対応していない可能性があります。
	
	\begin{itembox}[l]{出力結果}
		\directlua{tex.sprint(now())}
	\end{itembox}

	\subsection{動作を確認した環境}
	\begin{itemize}
		\item Linux Mint 19.3
		\item LuaTeX Version 1.10.0 (TeX Live 2019)
			\begin{itemize}
				\item Compiled with libpng 1.6.36; using 1.6.36
				\item Compiled with lua version 5.3.5
				\item Compiled with mplib version 2.00
				\item Compiled with zlib 1.2.11; using 1.2.11
			\end{itemize}
	\end{itemize}


	\section{二項係数の計算}
	\subsection{二項係数とは}
	二項係数とは、二項べき$(x + y)^{n}$を展開した時の$x^{k}y^{n-k}$の項の係数のことを言います。すなわち、

	\begin{equation}
		(x+y)^{n} = \sum_{k=0}^{n} {}_{n}\mathrm{C}_{k} x^{k}y^{n-k}
	\end{equation}

	における${}_n\mathrm{C}_{k}$のことである。

	\begin{equation}
		{}_{n}\mathrm{C}_{k} = \frac{n!}{k!(n-k)!}
	\end{equation}

	なお、$n = 0,1,2,3,\cdot$として、(1)の展開式のうち二項係数だけを上から順に書き出したものをパスカルの三角形と呼んでいる。今回は再帰を用いてこれを計算するスクリプトを書いた。これを以下に示す。

	\subsection{コードと実行結果}
	\begin{lstlisting}
-- パスカルの三角形を計算するスクリプト
function pascalTriangle(row, upto)
	tex.sprint(table.concat(row, " ")..n())
	if #row < upto then
		local i
		local tmp = {1}
		for i = 1, (#row - 1) do
			table.insert(tmp, row[i] + row[i+1])
		end 
		table.insert(tmp, 1)
		pascalTriangle(tmp, upto)
	end
end
	\end{lstlisting}

	\begin{itembox}[l]{出力結果}
		\centering
		{\scriptsize \directlua{pascalTriangle({1}, 19)}}
	\end{itembox}

	
	\section{モンテカルロ法による円周率の計算}
	\begin{tikzpicture}
		\draw[thick, -stealth](-1,0)--(6,0) node [anchor=north]{$x$};
		\draw[thick, -stealth](0,-1)--(0,6) node [anchor=east]{$y$};
		\draw[domain=0:5] plot(\x, {sqrt(25 - \x^2)});
		\node [anchor=north] at(3,3){$x^{2}+y^{2}=1$};
		\draw [thick](0,5) node [anchor=east]{$1$}--(5,5)--(5,0) node [anchor=north]{$1$};
		\node [anchor=south] at(5.5,5){$(1,1)$};
	\end{tikzpicture}

	\directlua{
		local result = calcPi(50000)
		tex.print("試行回数: " .. result[1] .. n())
		tex.print("円の中に入った回数: " .. result[2] .. n())
		tex.print("円周率: " .. result[3] .. n())
		tex.print("誤差: " .. 1 - math.pi / result[3] .. n())
	}
	
	\vspace{49pt}
	
	\directlua{
		local num = 999
		tex.sprint(num.."までの素数を赤文字で表示"..n())
		primeSeries(num)
	}

	\vspace{49pt}
	
	\newcommand{\fracinline}[2]{
		$\raisebox{0.4ex}{\small $#1$}
		\raisebox{0ex}{\large $/$}
		\raisebox{-0.2ex}{\small $#2$}$
	}

	\begin{tabular}{c||c|c|c|c}
		n & $a_{n}$ & $1-\fracinline{a_{n}}{\sqrt{1000}}$ & $b_{n}$ & $1-\fracinline{b_{n}}{\sqrt{1000}}$ \\
		\hline
		0 & 500 & - & 2 & - \\

	\directlua{
		for i=1,10 do
			local calc = calcSqrt(1000,i)
			r = math.sqrt(1000)
			tex.print(table.concat({i, calc[1], 1-(calc[1]/r), calc[2], 1-(calc[2]/r)}, " & ") .. " " .. n())
		end
	}
	
	\end{tabular}
	
	%\directlua{getRemoteImage("https://static.tenki.jp/static-images/radar/recent/japan-detail-middle.jpg")}
	
\end{document}
