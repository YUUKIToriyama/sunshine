\documentclass[a4paper]{article}
\usepackage{luatexja}
\usepackage{luacode}

\begin{luacode*}
	-- LaTeXの改行を出すための函数
	function n()
		return string.char(92)..string.char(92)
	end
	
	-- パスカルの三角形を計算するスクリプト
	function pascalTriangle(row, upto)
		tex.sprint(table.concat(row, " ")..n())
		if #row < upto then
			local i
			local tmp = {1}
			for i = 1, (#row - 1) do
				table.insert(tmp, row[i] + row[i+1])
			end 
			table.insert(tmp, 1)
			pascalTriangle(tmp, upto)
		end
	end

	-- 円周率を計算するスクリプト(モンテカルロ法)
	function calcPi(m)
		local count = 0
		for i = 1, m do
			x, y = math.random(), math.random()
			if x^2 + y^2 < 1 then
				count = count + 1
			end
		end
		return {m,count, 4 * count / m}
	end

\end{luacode*}

\begin{document}

\title{luaLaTeXであそぼう}
\author{YUUKIToriyama}
\date{}
\maketitle

	\centering
	\directlua{pascalTriangle({1}, 17)}	

	\vspace{49pt}

	\directlua{
		local result = calcPi(5000000)
		tex.print("試行回数: " .. result[1] .. n())
		tex.print("円の中に入った回数: " .. result[2] .. n())
		tex.print("円周率: " .. result[3] .. n())
		tex.print("誤差: " .. 1 - math.pi / result[3] .. n())
	}
	
	
\end{document}
